%!TEX root = ./template-skripsi.tex
%-------------------------------------------------------------------------------
% 								BAB I
% 							LATAR BELAKANG
%-------------------------------------------------------------------------------

\chapter{PENDAHULUAN}
\section{Latar Belakang Masalah}

Luka kronis adalah masalah kritis dalam kesehatan. Di Amerika Serikat, sekitar 6,5 juta orang menderita luka kronis dan biaya perawatan luka kronis menghabiskan sekitar \$20 miliar per tahun. Bahkan di negara maju, sekitar 1-2\% dari seluruh populasi terkena luka kronis selama hidup mereka. \citep{Biswas2018superpixel:1}. Luka kronis berdampak terhadap finansial dan penurunan kualitas hidup pasien. Kerusakan fisik, sosial, dan emosional seperti penurunan mobilitas, rasa sakit, ketidaknyamanan, membatasi kinerja aktivitas sehari-hari. Isolasi sosial, frustasi, dan reaksi psikologis lainnya yang menimbulkan dampak pada kehidupan pasien. \citep{NaiaraVogt2020quality:2}. Di Indonesia sendiri pengidap luka kronis berjumlah sekitar 24\% dari 8,6\% total populasi terhadap kasus diabetes. \citep{Safitri2022hubungan:3}. 

Pada penelitian sebelumnya yang dilakukan oleh Salsa yang berjudul "Rancang Bangun Aplikasi dan \emph{Web Service} Pengkajian Luka Kronis Khusus Modul Pengolahan Citra Berbasis Android". Salsa melakukan wawancara dengan Ratna Aryani, M.Kep., Dosen Politeknik Negeri Jakarta I, diperoleh bahwa saat melakukan penggantian balutan luka dan pengecekan awal kondisi luka dilakukan pengkajian luka. Berikut  langkah-langkah pengkajian luka diawali dengan balutan luka dibuka, lalu luka dicuci, dan diakhiri dengan proses pengkajian luka. Instrumen yang dipilih saat melakukan pengkajian luka ialah Bates-Jensen \emph{Wound Assesment Tools} (BWAT). Pada BWAT ada 13 kategori penilaian yakni beberapa di antaranya  tepi luka, ukuran luka, epitalisasi dan jumlah eksudat (cairan tubuh yang keluar dari jaringan selama peradangan). Saat ini data pengkajian luka masih dilakukan secara tradisional dicatat dalam arsip atau catatan kertas, maka dari itu salsa mengusulkan untuk mendigitalisasi pencatatan data luka yang sudah dikaji. \citep{Rahmadati2023rancang:4}.

Penelitian lain yang terkait juga dilakukan oleh Ardiansyah, menjelaskan bahwa Rumah Sakit Umum Kambang Jambi masih memakai cara tradisional dalam pelayanannya seperti mendapatkan nomor antrian berobat, informasi mengenai jadwal dokter dan jumlah seluruh pasien. Hal ini menyebabkan pemanfaatan informasi menjadi kurang maksimal, berjalan kurang efektif dan lama dalam prosesnya. Dengan adanya permasalahan yang terjadi ardiansyah dan kawannya menyimpulkan bahwa dibutuhkannya Sistem Informasi Manajemen Rumah Sakit Berbasis \emph{Website} untuk menyelesaikan permasalahan yang ada. Sehingga meminimalkan kekurangan dan ketidak efektifan dalam pelayanan. \citep{Ardiansyah2021analisis:10}.

Dalam jurnal berjudul "Sistem Informasi Rekam Medis Pada Rumah Sakit Umum Daerah (RSUD) Pacitan Berbasis \emph{Web Base}" oleh Gunawan Susanto. Pencatatan riwayat dan data rekam medis kesehatan milik pasien merupakan hal yang krusial dalam dunia medis karena data tersebut digunakan untuk pemeriksaan pasien selanjutnya. Sistem pencatatan yang dipakai memiliki kelemahan. Hal ini dikarenakan data rekam medis pasien hanya disimpan secara lokal di tempat pasien diperiksa dan dirawat serta pertukaran data langsung antara divisi medis tidak diperbolehkan. Maka dari itu dilakukan pengembangan sistem informasi rekam medis yang memiliki tujuan untuk menyelesaikan kelemahan yang dimiliki oleh sistem pencatatan rekam medis pasien yang sebelumnya, yaitu alternatif teknologi yang dapat diterapkan di masa yang akan datang untuk pencatatan dan penyampaian data rekam medis. \citep{Gunawan2011sistem:11}.

Pada penelitian yang dilaksanakan oleh Inah Carminah yang berjudul "Aplikasi Monitoring Perawatan Luka Diabetes Melitus Berbasis \emph{Website}". Proses pelayanan yang masih menggunakan \emph{paper base system} memiliki risiko kerusakan atau kehilangan data rekam medis pasien. Selain itu membuat perawat kewalahan ketika mencari data rekam medis pasien secara satu-persatu ketika dibutuhkan ketika pasien datang untuk berobat kembali. Berangkat dari permasalahan di atas memotivasi instansi untuk membuat apikasi dengan tujuan untuk meningkatkan kualitas pelayanan pasien saat berobat. \citep{Carminah2021aplikasi:12}.

%
Didalam buku berjudul “Rancang Bangun Aplikasi \emph{Mobile} Android Sebagai Alat Deteksi Warna Dasar Luka Dalam Membantu Proses Pengkajian Luka Kronis Dengan Nekrosis”, Tehnik pengkajian luka berdasarkan warna luka yang umum digunakan salah satunya The RYB (\emph{Red-Yellow-Black}) \emph{wound classification system}. Metode ini digunakan dengan mengandalkan subyektifitas dari perawat luka. Hasil penelitian pada buku ini menunjukkan bahwa perawat mampu mengetahui perbedaan warna luka secara otomatis yang membantu proses pengkajian luka kronis dengan nekrosis. \citep{Aryani2018rancang:6}. Ia juga meneliti dan menemukan bahwa perban basah membantu mempercepat proses penyembuhan luka. Perawat harus mempertimbangkan untuk menggunakan balutan basah daripada perawatan standar untuk meningkatkan penyembuhan. Namun, perawat harus melindungi luka dari kelembapan yang berlebihan karena dapat merusak kulit di sekitar luka atau di dalam luka. \citep{Aryani2016accelerating:7}.

%
Pada payung penelitian \emph{medical imaging} yang sama dengan peneliti juga sudah pernah dilakukan penelitian mengenai Pengaruh Penggunaan \emph{Color Model} LAB dalam Kalibrasi Warna Luka Menggunakan Metode Segmentasi K-\emph{Means} dan \emph{Mean Shift} oleh rekan sesama peneliti. \citep{Khairunnisa2021pengaruh:8}. Dan Muhamad rizki juga melakukan penelitian deteksi tepi luka menggunakan metode \emph{Active Contour} yang ditambah interpolasi. \citep{Rizki2022deteksi:9}. Kedua penelitian tersebut merupakan penelitian berdasarkan dua kategori pengkajian luka yaitu warna luka dan tepi luka, algoritma yang dikembangkan pada penelitian tersebut direncanakan akan terintegrasi dalam satu ekosistem aplikasi, yakni sistem informasi keperawatan luka. Dimana pada penelitian Salsa Rahmadati melakukan perancangan aplikasi pengkajian luka kronis berbasis Android sesuai modul \emph{image processing}. \citep{Rahmadati2023rancang:4}. 

%
Berdasarkan hal di atas, penulis tertarik untuk melakukan penelitian yang bertujuan untuk membuat sistem informasi keperawatan luka dengan dasar pengembangan menggunakan data paparan presentasi bersama ibu Irma Puspita Arisanti selaku pemilik klinik \emph{moist care} dan sesuai dengan proposal PKM-PI dengan judul "Pengembangan Pelayanan Sistem Informasi Klinik Serta Fitur Keperawatan Luka Pada Aplikasi Untuk Mendukung Integrasi Data Kesehatan Dan Ketahanan Nasional Bidang Kesehatan" yang dibuat oleh Hafiz dan tim. Melanjutkan penelitian sebelumnya yang dilakukan oleh Salsa, dimana pengkajian luka masih dilakukan dengan cara manual atau arsip kertas sehingga Salsa membuat aplikasi untuk mengarsipkan data secara digital dan peneliti mengembangkan \emph{web} aplikasi yang berkaitan dengan aplikasi sebelumnya untuk dapat diakses datanya oleh klinik dengan maksud seluruh staff klinik yang berkepentingan dapat dengan mudah mengaksesnya.
 
Sistem Informasi tersebut diharapakan dapat menambah opsi pendaftaran berobat secara \emph{online} selain daripada pendaftaran secara \emph{offline}, membantu pengelolaan antrian, membantu integrasi data pasien dan perawat secara digital, manajemen invetaris, beserta verifikasi dan validasi biaya tagihan sehingga dapat mempermudah pelayanan.  

\section{Rumusan Masalah}
Berdasarkan uraian pada latar belakang yang diutarakan di atas, maka perumusan masalah pada penelitian ini adalah bagaimana membuat rancang bangun sistem informasi keperawatan luka?

\section{Pembatasan Masalah}
Batasan masalah dalam penelitian ini adalah:

\begin{enumerate}
	\item Sistem informasi keperawatan luka dibuat dengan dasar instrumen pengkajian Bates-Jensen \emph{Wound Assessment Tool} (BWAT).
	
	\item \emph{User} aplikasi sistem informasi keperawatan luka adalah perawat dan admin klinik.
	
	\item Sistem informasi keperawatan luka dibuat berbasis \emph{Website}
	
	\item Sistem informasi keperawatan luka dibuat berdasarkan paparan presentasi bersama pemilik klinik \emph{Moist Care} yaitu ibu Irma Puspita Arisanti.
	
	\item Model pengembangan yang digunakan untuk mengembangkan sistem informasi keperawatan luka adalah \emph{scrum}.
	
	\item Fitur-fitur yang diimplementasi pada sistem informasi keperawatan luka, diantaranya adalah pembuatan akun pasien, \emph{dashboard} klinik, pemeriksaan kesehatan dan sebagian proses pengobatan luka (\emph{view} dan \emph{web service} inventaris dan layanan).
	
\end{enumerate}

\section{Tujuan Penelitian}
Penelitian ini bertujuan untuk membuat rancang bangun sistem informasi keperawatan luka di klinik \emph{Moist Care}.

\section{Manfaat Penelitian}
\begin{enumerate}
	\item Bagi penulis
	
	Penelitian yang dilakukan merupakan media penerapan dari berbagai ilmu pengetahuan, khususnya dalam perancangan sistem informasi keperawatan luka pada klinik \emph{Moist Care}.
	
	\item Bagi Program Studi Ilmu Komputer
	
	Penelitian ini dapat menjadi pintu gerbang untuk penelitian selanjutnya di masa depan.
	
	\item Bagi Universitas Negeri Jakarta
	
	Menjadi evaluasi akademik program studi Ilmu Komputer dalam penulisan skripsi sehingga dapat meningkatkan kualitas pendidikan program studi Ilmu Komputer di Universitas Negeri Jakarta.
	
\end{enumerate}

% Baris ini digunakan untuk membantu dalam melakukan sitasi
% Karena diapit dengan comment, maka baris ini akan diabaikan
% oleh compiler LaTeX.
\begin{comment}
\bibliography{daftar-pustaka}
\end{comment}
