%!TEX root = ./template-skripsi.tex
%-------------------------------------------------------------------------------
%                            	BAB IV
%               		KESIMPULAN DAN SARAN
%-------------------------------------------------------------------------------

\chapter{KESIMPULAN DAN SARAN}

\section{Kesimpulan}

Berdasarkan hasil implementasi sistem informasi keperawatan luka, maka dapat diambil kesimpulan sebagai berikut:

\begin{enumerate}
	\item Terciptanya sistem informasi keperawatan luka yang sudah mengimplementasikan sebagian fitur-fitur pada \emph{product backlog}. Adapun perancangannya dilakukan dengan metode \emph{scrum} dengan tahapan penyusunan \emph{product backlog}, \emph{sprint backlog} dan terbagi menjadi empat \emph{sprint}.
	\item Fitur pada sistem yang selesai dibuat diantaranya adalah pembuatan akun pasien, \emph{dashboard} klinik, pemeriksaan kesehatan dan sebagian proses pengobatan luka (\emph{view} dan \emph{web service} inventaris dan layanan).
	\item Fitur pada sistem yang tidak selesai dibuat diantaranya adalah sebagian proses pengobatan luka (\emph{view} dan \emph{web service} hasil pengkajian luka dan \emph{database} foto luka yang dikategorisasikan berdasarkan perawat), pendaftaran pasien berobat, pengelolaan antrian, dan administrasi keuangan.
	\item Terimplementasikannya \emph{web service} yang berfungsi sebagai \emph{back-end} sistem informasi keperawatan luka.
	\item Sistem informasi keperawatan luka dikembangkan menggunakan bahasa \emph{python} dengan bantuan \emph{framework flask}.  
\end{enumerate}

\section{Saran}

Adapun beberapa saran untuk penelitian selanjutnya adalah sistem informasi keperawatan luka dapat dilanjutkan pengembangannya untuk fitur-fitur yang belum terselesaikan atau menambahkan fitur-fitur lain yang dibutuhkan pada masa yang akan datang.

% Baris ini digunakan untuk membantu dalam melakukan sitasi
% Karena diapit dengan comment, maka baris ini akan diabaikan
% oleh compiler LaTeX.
\begin{comment}
\bibliography{daftar-pustaka}
\end{comment}